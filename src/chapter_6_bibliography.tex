%%%%%%%%%%%%%%%%%%%%%%%%%%%%%%%%%%%%%%%%%%
% CONTRIBUTION TO THE SURFEX DOCUMENTATION
% Author        : P. Le Moigne
% Original      : January 2009
% Last Update   : July    2022
%%%%%%%%%%%%%%%%%%%%%%%%%%%%%%%%%%%%%%%%%%

\chapter{References}
\minitoc

%%%%%%%%%%%%%%%%%%%%%%%%%%%%%%%%%%%%%%%%%%%%%%%%%%%
\section{Isba}
\begin{description}

\item
Anderson, E. A., 1976:
A point energy and mass balance model of a snow cover.
{\it NOAA Tech. Rep. NWS 19}, 150 pp. U.S. Dept. of
Commer., Washington, D.C.

\item
Beven KJ, Kirkby MJ (1979) A physically-based variable contributing area model of basin 
hydrology. Hydrol. Sci. Bull. 24: 43-69.

\item
Best, M.J., A. Beljaars, J. Polcher, P. Viterbo, 2004:
A proposed structure for coupling tiled surfaces with the planetary boundary layer.
{\em Journal of Hydrometeorology}, {\bf 5}, 1271-1278.

\item
Bhumralkar, C.M., 1975:
Numerical experiment on the computation of ground surface
temperature in an atmospheric general circulation model.
{\em J. Appl. Meteor.}, {\bf 14}, 1246-1258.

\item
Blackadar, A.K., 1976:
Modeling the nocturnal boundary layer.
{\em Proc. Third Symp. on Atmospheric Turbulence,
Diffusion and Air Quality }, Boston, Amer. Meteor. Soc., 46-49.

\item
Boone, A., Modelisation des processus hydrologiques dans le schema de surface ISBA: Inclusion d'un reservoir hydrologique, du gel et modelisation de la neige. PhD thesis, University Paul Sabatier, TOULOUSE, France, 2000. 252pp.

\item
Boone, A.,
and P. Etchevers, 2000:
An intercomparison of three snow schemes of varying complexity
coupled to the same land-surface and macroscale hydrologic models.
{\em J. Hydrometeor.},


\item
Boone, A.,
J.-C. Calvet and J. Noilhan, 1999:
Inclusion of a third soil layer in a
land-surface scheme using the force-restore method,
{\em J. Appl. Meteor.},
{\bf 38},
1611-1630.

\item
Boone, A.,
V. Masson, T. Meyers, and J. Noilhan, 2000:
The influence of the inclusion of soil freezing
on simulations by a soil-atmosphere-transfer scheme.
{\em J. Appl. Meteor.},
(in press).

\item
Braud, I., J. Noilhan, P. Bessemoulin, P. Mascart, R. Haverkamp,
and M. Vauclin, 1993:
Bare-ground surface heat and water exchanges under dry conditions:
Observations and parameterization.
{\em Bound.-Layer Meteror.},
{\bf 66},
173-200.

\item
Caballero, Y., Voirin-Morel, S., Habets, F., Noilhan, J., LeMoigne, P., Lehenaff, A., and Boone,
A.: Hydrological sensitivity of the Adour-Garonne river basin to climate change, Water. Resour. Res., 43, W07448, doi:10.1029/2005WR004192, 2007

\item
Clapp R, Hornberger G (1978) Empirical equations for some soil hydraulic properties. Wat. 
Resour. Res. 14: 601-604.

\item
Deardorff, J.W., 1978:
Efficient prediction of ground surface temperature and moisture
with inclusion of a layer of vegetation.
{\em J. Geophys. Res.},
{\bf 20},
1889-1903.

\item
Deardorff, J.W., 1977:
A parameterization of ground surface moisture content for
use in atmospheric prediction models.
{\em J. Appl. Meteor.},
{\bf 16},
1182-1185.

\item
Decharme B., H. Douville, A. Boone, F. Habets, and J. Noilhan, 2006: Impact of an 
exponential profile of saturated hydraulic conductivity within the ISBA LSM: simulations 
over the Rhône basin. J. Hydrometeor, 7, 61-80.

\item
Decharme B. and H. Douville, 2006: Introduction of a sub-grid hydrology in the ISBA land 
surface model. Climate Dyn., 26, 65 - 78.

\item
Decharme B. and H. Douville, 2007: Global validation of the ISBA Sub-Grid Hydrology. 
Climate Dyn., 29, 21-37.

\item
Dickinson, R.E., 1984:
Modeling evapotranspiration for three dimensional global
climate models.
{\em Climate Processes and Climate Sensitivity.
Geophys. Monogr.},
{\bf 29},
58-72.

\item
Douville, H., 1994:
D\'eveloppement et validation locale d'une nouvelle
param\'etrisation du manteau neigeux.
Note 36 GMME/M\'et\'eo-France.

\item
Douville, H., J.-F. Royer, and J.-F. Mahfouf,1995:
A new snow parameterization for the French community climate
model.  Part I:  Validation in stand-alone experiments.
{\em Climate Dyn.},
submitted.

\item
Fan Y, Wood EF, Baeck ML, Smith JA (1996) The fractional coverage of rainfall over a grid: 
Analyses of NEXRAD data over the southern plains. Water Resour. Res. 32: 2787-2802.
Habets F, Saulnier GM (2001) Sub-grid runoff parameterization. Phys. Chem. Earth 26: 455-
459.

\item
Giard, D., and E. Bazile, 1999:
Implementation of a new assimilation scheme for
soil and surface variables in a global NWP model.
{\em Mon. Wea. Rev.},
(in press).

\item
Giordani, H., 1993:
Exp\'eriences de validation unidimensionnelles du sch\'ema
de surface NP89 aux normes Arp\`ege sur trois sites de la
campagne EFEDA 91.
Note de travail 24 GMME/M\'et\'eo-France.

\item
Giordani, H., J. Noilhan, P. Lacarrere, P. Bessemoulin, and P. Mascart, 1996 : Modelling the surface
processes and the atmospheric boundary layer for semi-arid conditions. Agricultural and Forest
Meteorology, 80, 263–287.

\item
HABETS F., NOILHAN J., GOLAZ C., GOUTORBE J.P., LACARRERE P., LEBLOIS E., LEDOUX E., MARTIN E., OTTLE C., VIDAL-MADJAR D., 1999. The ISBA surface scheme in a macroscale hydrological model applied to the Hapex-Mobilhy area. Part I : model and data base, Journal of Hydrology, 217, p.75-96.

\item
Habets F. and Saulnier G.-M.,Subgrid runoff parameterization.
Physics and Chemistry of the Earth, Part B: Hydrology, Oceans and Atmosphere
Volume 26, Issues 5-6, 2001, Pages 455-459 

\item
Jacquemin, B., and J. Noilhan, 1990:
Validation of a land surface parameterization using the
HAPEX-MOBILHY data set.
{\em Bound.-Layer Meteor.},
{\bf 52},
93-134.

\item
Johnsson, H., and L.-C. Lundin, 1991:
Surface runoff and soil water percolation as affected by snow
and soil frost.
{\em J. Hydro.},
{\bf 122},
141-158.

\item
Loth, B.,
H.-F. Graf, and J. M. Oberhuber, 1993:
Snow cover model for global climate simulations.
{\em J. of Geophys. Res.},
{\bf 98},
10451-10464.

\item
Louis, J.F., 1979:
A parametric model of vertical eddy fluxes in the atmosphere.
{\em Bound.-Layer Meteor.},
{\bf 17},
187-202.

\item
Lynch-Stieglitz, M., 1994: The development and validation
of a simple snow model for the GISS GCM.
{\em J. Clim.},
{\bf 7},
1842-1855.

\item
Mahfouf, J.-F., J. Noilhan, and P. P\'eris, 1994:
Simulations du bilan hydrique avec ISBA:  Application
au cycle annuel dans le cadre de PILPS.
Atelier de mod\'elisation de l'atmosph\`ere,
CNRM/M\'et\'eo-France,
December 1994, Toulouse, France, 83-92.

\item
Mahfouf, J.-F., and J. Noilhan, 1991:
Comparative study of various formulations of evaporation
from bare soil using in situ data.
{\em J. Appl. Meteor.}, {\bf 9}, 1354-1365.

\item
Mahfouf, J.-F. and Noilhan J., 1996: Inclusion of gravitational drainage in a land surface scheme based on the force restore method. Journal of Applied Meteorology, 35, 987-992.

\item
Mascart, P., J. Noilhan, and H. Giordani, 1995:
A modified parameterization of flux-profile relationships
in the surface layer using different roughness length
values for heat and momentum.
{\em Bound.-Layer Meteor.},{\bf 72}, 331-344.

\item
Masson, 2000:
A physically-based scheme for the urban energy budget in atmospheric models.
{\it Boundary Layer Meteorology}, in press.

\item
Masson V., J.-L. Champeaux, F. Chauvin, C. Meriguet and R. Lacaze, 2003 : 
A global database of land surface parameters at 1km resolution in meteorological and climate models.
{\it J. Climate},{\bf 16}, 1261-1282.

\item
Noilhan, J., and P. Lacarr\`ere, 1995:
GCM grid-scale evaporation from mesoscale modeling.
{\em J. Climate},
in press.

\item
Noilhan, J., and S. Planton, 1989:
A simple parameterization of land surface processes for
meteorological models.
{\em Mon. Wea. Rev.}, {\bf 117}, 536-549.

\item
Noilhan J. and Mahfouf J.-F., 1996: The ISBA land surface parameterisation scheme Global and Planetary Change, 13, 145-159.

\item
Polcher, J., B. McAveney, P. Viterbo, M.-A. Gaertner, A. Hahmann, J.-F. Mahfouf, J. Noilhan, T. Phillips, A. Pitman, C.A., Schlosser, J.-P. Schulz, B. Timbal, D. Verseghy and Y. Xue, 1998:
A proposal for a general interface between land-surface schemes and
general circulation models.
{\em Global and Planetary Change}, {\bf 19}, 261-276.

\item
Silvapalan M, Beven KJ, Wood EF (1987) On hydrologic similarity: 2. A scaled model of 
storm runoff production, Water Resour. Res. 23: 2266-2278.

\item
Sun, S.,
J. Jin, and Y. Xue, 1999:
A simple snow-atmosphere-soil transfer (SAST) model.
{\em J. of Geophys. Res.},
{\bf 104},
19587-19579.

\item
Verseghy, D., 1991:
CLASS - A Canadian land surface scheme for GCMs.
I:  Soil model.
{\em Int. J. Climatol.},
{\bf 11},
111-133.

\end{description}

%%%%%%%%%%%%%%%%%%%%%%%%%%%%%%%%%%%%%%%%%%%%%%%%%%%
\section{Isba-A-gs}
\begin{description}

\item
Calvet, J.-C., J. Noilhan, J.-L. Roujean, P. Bessemoulin, M. Cabelguenne, A. Olioso, and J.-P.
Wigneron (1998), An interactive vegetation SVAT model tested against data from six contrasting
sites, Agric. For. Meteorol., 92, 73-95.
\item
Calvet, J.-C. (2000), Investigating soil and atmospheric plant water stress using physiological and
micrometeorological data, Agric. For. Meteorol., 103, 229-247.
\item
Calvet, J.-C., and J.-F. Soussana (2001), Modeling CO2-enrichment effects using an interactive
vegetation SVAT scheme, Agric. For. Meteorol., 108, 129-152.
\item
Calvet, J.-C., V. Rivalland, C. Picon-Cochard, and J.-M. Guehl (2004), Modelling forest transpiration
and CO2 fluxes - response to soil moisture stress, Agric. For. Meteorol., 124, 143-156.
\item
Calvet, J.-C., A.-L. Gibelin, E. Martin, P. Le Moigne, H. Douville, J. Noilhan (2008), Past and future
scenarios of the effect of carbon dioxide on plant growth and transpiration for three vegetation
types of south-western France, Atmos. Chem. Phys., 8, 397–406.
\item
Collatz, G. J., M., Ribas-Carbo, J. A., Berry (1992), Coupled photosynthesis-stomatal conductance
model for leaves of C4 plants. Aust. J. Plant Physiol. 19, 519–538.
GSWP2 (2002) Science and implementation plan gswp2, IGPO Publication Series, 37.
\item
Farquhar, G. D., S., von Caemmerer, J. A., Berry (1980), A biochemical model of photosynthetic CO2
assimilation in leaves of C3 species. Planta, 149, 78–90.
\item
Gibelin, A.-L., J.-C. Calvet, J.-L. Roujean, L. Jarlan, S. O. Los (2006), Ability of the land surface model
ISBA-A-gs to simulate leaf area index at the global scale: comparison with satellites products.
J. Geophys. Res. 111, D18102.
\item
Gibelin, A.-L., J.-C. Calvet, and N. Viovy (2008), Modelling energy and CO2 fluxes with an interactive
vegetation land surface model – Evaluation at high and middle latitudes, Agric. For. Meteorol.,
148, 1611-1628, doi: 10.1016/j.agrformet.2008.05.013.
\item
Goudriaan, J., H.H. van Laar, H. Van Keulen, and W. Louwerse (1985), Photosynthesis, CO2 and
plant production. In: W. Day and R.K. Atkin (Eds.), Wheat growth and modelling. NATO ASI Series,
Plenum Press, New York, Series A, 86, 107-122.
\item
Jacobs, C. M. J. (1994), Direct impact of CO2 enrichment on regional transpiration. Ph. D. Thesis,
Agricultural University, Wageningen
\item
Jacobs, C. M. J., B. J. J. M. Van den Hurk, and H. A. R. De Bruin (1996), Stomatal behaviour and
photosynthetic rate of unstressed grapevines in semi-arid conditions, Agric. For. Meteorol., 80,
111-134.
\item
Lafont S., A. Beljaars, M. Voogt, L. Jarlan, P. Viterbo, B. van den Hurk, J.-C. Calvet (2006),
Comparison of C-TESSEL CO2 fluxes with TransCom CO2 fluxes. Proc. Second Recent Advances
in Quantitative Remote Sensing II, Torrent (Valencia), Spain, 26-29 September 2006, 474-477.
\item
Lemaire, G. and F. Gastal (1997), N uptake and distribution in plant canopies. In: Lemaire, G. (Ed.),
Diagnosis of the Nitrogen Status in Crops. Springer, Berlin, pp. 343.
\item
Martin, E., Le Moigne, P., Masson, V., et al. (2007), Le code de surface externalis\'ee SURFEX de
M\'et\'eo- France, Ateliers de Mod\'elisation de l’Atmosph\`ere (http://www.cnrm.meteo.fr/ama2007/),
Toulouse, 16–18 January, 2007.
\item
Masson, V., J.-L. Champeaux, F. Chauvin, C. Meriguet and R. Lacaze (2003), A global database of
land surface parameters at 1-km resolution in meteorological and climate models, J. Climate, 16,
1261-1282.
\item
Norman, J.M., R. Garcia, S.B. Verma (1992), Soil surface CO2 fluxes and the carbon budget of a
grassland, J. Geophys. Res., 97(D17), 18845-18853.
\item
Rivalland, V., J.-C. Calvet, P. Berbigier, Y. Brunet, A. Granier (2005), Transpiration and CO2 fluxes of
a pine forest: modelling the undergrowth effect, Ann. Geophys., 23, pp 291-304.
\item
Roujean, J.-L. (1996), A tractable physical model of shortwave radiation interception by vegetative
canopies. J. Geophys. Res., 101D5, 9523-9532.
\item
Van den Hurk, B. J. J. M., P. Viterbo, A. C. M. Beljaars, and A. K. Betts (2000), Offine validation of the
ERA40 surface scheme, ECMWF TechMemo. 295, 42 pp., ECMWF, Reading.
\item
Voogt, M., B.J.J.M. van den Hurk and C. Jacobs (2006), The ECMWF land surface scheme extended
with a photosynthesis and LAI module tested for a coniferous site, KNMI publication: WR-06-02,
3/1/2006, 22 pp, De Bilt, The Netherlands,
http://www.knmi.nl/publications/fulltexts/agrformetd0600207.pdf .
\item
Yin, X. (2002), Responses of leaf nitrogen concentration and specific leaf area to atmospheric CO2
enrichment: a retrospective synthesis across 62 species, Global Change Biol., 8(7), 631–642.


\end{description}

%%%%%%%%%%%%%%%%%%%%%%%%%%%%%%%%%%%%%%%%%%%%%%%%%%%
\section{Teb}
\begin{description}

\item
Arnfield  A.J. and Grimmond C.S.B.. An urban energy budget Model and its application to Urban
Storage Heat Flux modeling. Energy and buildings, 27:61–68, 1998.
\item
Arya  S.P. Introduction to Micrometeorology. Academic Press, Inc., 1988.
\item
 Best M.J. A model to predict surface temperatures. Boundary-Layer Meteorol., 88:279–306, 1998.
\item
Bottema M. Urban Roughness modelling in relation to Pollutant Dispersion. Atmos. Environ.,
31(18):3059–3075, 1997.
\item
Deardorff J.W. Efficient prediction of ground temperature and moisture with inclusion of a layer of
vegetation. J. Geophys. Res., 83:1889–1903, 1978.
\item
Feigenwinter C., Vogt R., and Parlow E. Vertical structure of selected turbulence characteristics above
an urban canopy. Theor. Appl. Climatol., 62:51–63, 1999.
\item
Grimmond C.S.B. and Oke T.R. Aerodynamic Properties of urban areas derived from Analysis of
Surface form. J. Appl. Meteorol., 38:1262–1292, 1999b.
\item
 Grimmond C.S.B. and Oke T.R. An evapotranspiration-interception Model for Urban areas. Water
Resour. Res., 27:1739–1755, 1991.
\item
 Grimmond C.S.B. and Oke T.R. Heat storage in urban areas : Local-scale observations and evaluation
of a simple model. J. Appl. Meteorol., 38:922–940, 1999a.
\item
 Grimmond C.S.B., Cleugh H.A. and Oke T.R. An objective Urban Heat Storage model and its comparison
with other Schemes. Atmos. Environ., 25B:311–326, 1991.
\item
Johnson G.T., Oke T.R., Lyons T.J., Steyn D.G., Watson I.D., and Voogt J.A. Simulation of
surface urban heat islands under ’ideal’ conditions at night. part i: theory and tests against field data.
Boundary-Layer Meteorol., 56:275–294, 1991.
\item
Lemonsu, A.; Grimmond, C. S. B. and Masson, V. Modeling the surface energy balance of the core of an old mediterranean city: Marseille. J. Appl. Meteorol., 2004 , 43 , 312-327
\item
Lemonsu, A. and Masson, V. Simulation of a summer urban breeze over Paris Boundary-Layer Meteorol., 2002 , 104 , 463-490
\item
Lemonsu, A.; Pigeon, G.; Masson, V. and Moppert, C. Sea-town interactions over Marseille: 3D urban boundary layer and thermodynamic fields near the surface Theor. and Appl. Climatol., 2006 , 84 , 171-178 

\item
 Mascart P., Noilhan J., and Giordani H. A modified parameterization of flux-profile relationship in
the surface layer using different roughness length values for heat and momentum. Boundary-Layer
Meteorol., 72:331–344, 1995.
\item
 Menut L. Etude exp´erimentale et th´eorique de la couche limite Atmosph´erique en agglom´eration
parisienne (experimental and theoretical study of the ABL in Paris area). PhD thesis, University
Pierre et Marie Curie, PARIS, France, 1997. 200pp.
\item
 Mills G.M. Simulation of the energy budget of an urban canyon-i. model structure and sensitivity
test. Atmos. Environ., 27B(2):157–170, 1993.
\item
 Noilhan J. A Model for the Net Total Radiation flux at the Surfaces of a Building. Building and
Environment, 16(4):259–266, 1981.
\item
 Noilhan J. and Planton S. A simple parameterization of land surface processes for meteorological
models. Mon. Wea. Rev., 117:536–549, 1989.
\item
Offerle, B.; Grimmond, C. S. B. and Fortuniak, K. Heat storage and anthropogenic heat flux in relation to the energy balance of a central European city centre Int. J. Climatol., 2005 , 25 , 1405-1419 

\item
 Oke, T.R. Boundary layer climates. 2nd edition. Methuen, London, 1987. 435pp.
\item
 Oke, T.R. The urban energy balance. Prog. Phys. Geogr., 12:471–508, 1988.
\item
 Oke, T.R., R.A. Spronken-Smith, E. J´auregui, and C.S.B. Grimmond. The energy balance of central
Mexico City during the dry season. Atmos. Environ., 33:3919–3930, 1999.
\item
 Peterseni R.L. A wind tunnel evaluation of methods for estimating surface roughness length at industrial
facilities. Atmos. Environ., 31(1):45–57, 1997.
\item
Pigeon, G.; Lemonsu, A.; Long, N.; Barrié, J.; Durand, P. and Masson, V. Urban thermodynamic island in a coastal city analyzed from an optimized surface network Boundary-Layer Meteorol., 2006 , 120 , 315-351

\item
Pigeon, G.; Lemonsu, A.; Grimmond, C.; Durand, P.; Thouron, O. and Masson, V. Divergence of turbulent fluxes in the surface layer: case of a coastal city Boundary-Layer Meteorol., 2007 , 124 , 269-290

\item
Pigeon, G.; Moscicki, M. A.; Voogt, J. A. and Masson, V. Simulation of fall and winter surface energy balance over a dense urban area using the TEB scheme Meteorology and Atmospheric Physics, 2008, vol 102, 159-172


\item
 Richards K. and Oke T.R. Dew in urban environments. Procedings of IInd AMS Urban Environment
Symposium, 1998.
\item
Roberts, S. M.; Oke, T. R.; Grimmond, C. S. B. and Voogt, J. A. Comparison of four methods to estimate urban heat storage JAMC, 2006 , 45 , 1766-1781 

\item
 Rotach M.W. Profiles of turbulence statistics in and above an urban street canyon. Atmos. Environ.,
29(13):1473–1486, 1995.
\item
 Roth M. Turbulent transfert: relationships over an urban surface. ii: Integral statistics. Quart. J. Roy.
Meteor. Soc., 119:1105–1120, 1993.
\item
 Roth M. and Oke T.R. Turbulent transfert: relationships over an urban surface. ii: spectral characteristics.
Quart. J. Roy. Meteor. Soc., 119:1071–1104, 1993.
\item
 Rowley F.B., Algren A.B., and Blackshaw J.L. Surface conductances as affected by air velocity,
temperature and character of surface. ASHRAE Trans., 36:429–446, 1930.
\item
 Rowley F.B. and Eckley W.A. Surface coefficients as affected by wind direction. ASHRAE Trans.,
38:33–46, 1932.
\item
Sarrat, C.; Lemonsu, A.; Masson, V. and Guedalia, D. Impact of urban heat island on regional atmospheric pollution Atmospheric Environment, 2006 , 40 , 1743-1758

\item
 Seaman N.L., Ludwig F., Donall E.G., Warner T.T., and Bhumralkar C.M. Numerical studies
of urban planetary boundary-layer structure under realistic synoptic conditions. J. Appl. Meteorol.,
28:760–781, 1989.
\item
 Soux A., Oke T.R., and Voogt J.A. Modelling and remote sensing of the urban surface. Procedings
of IInd AMS Urban Environment Symposium, 1998.
\item
 Sturrock N.S. and Cole R.J. The convective heat exchange at the external surface of buildings. Building
and environment, 12:207–214, 1977.
\item
 Taha H. Modifying a Mesoscale Meteorological Model to better incorporate Urban Heat Storage : A
bulk parametrerization approach. J. Appl. Meteorol., 38:466–473, 1999.
\item
 Terjung W.H. and O’Rourke P.A. Influences of physical structures on urban energy budgets.
Boundary-Layer Meteorol., 19:421–439, 1980.
\item
 Wieringa J. Representative roughness parameters for homogeneous terrain. Boundary-Layer Meteorol.,
63:323–363, 1993.

\end{description}

%%%%%%%%%%%%%%%%%%%%%%%%%%%%%%%%%%%%%%%%%%%%%%%%%%%
%%%%%%%%%%%%%%%%%%%%%%%%%%%%%%%%%%%%%%%%%%%%%%%%%%%
\section{Surface Boundary Layer scheme}
\begin{description}

\item
Belcher, S. E., N. Jerram, and J. C. R. Hunt, 2003: Adjustment of a turbulent boundary layer to
a canopy of roughness elements. Journal of Fluid Mechanics, 488, 369–398.
\item
Beljaars, A. C. M. and A. A. M. Holtslag, 1991: Flux parameterization over land surfaces for
atmospheric models. J. Appl. Meteorol., 30, 327–341.
\item
Best, M. J., A. Beljaars, J. Polcher, and P. Viterbo, 2004: A proposed structure for coupling tiled
surfaces with the planetary boundary layer. Journal of hydrometeorology, 5, 1271–1278.
\item
Bougeault, P., B. Bret, P. Lacarr`ere, and J.Noilhan, 1988: Design and implementation of a
land surface processes parameterization in a mesoscale model. Parameterization of fluxes
over land surface, European Center of Medium-Range Weather Forecasting, Shinfield Park,
Reading RG2 9AX, U.K., 95–120.
\item
Bubnov´a, R., G. Hello, P. B´enard, and J.-F. Geleyn, 1995: Integration of the fully elastic equations
cast in the hydrostatic pressure terrain-following coordinate in the framework of the
ARPEGE/ALADIN NWP system. Mon. Wea. Rev., 123, 515–535.
\item
Chen, T. H. and coauthors, 1997: Cabauw experimental results from the project for intercomparison
of land-surface parameterization schemes. Journal of Climate, 10, 1194–1215.
\item
Cheng, Y., V. Canuto, and A. Howard, 2002: An improved model for the turbulent pbl. J. Atmos.
Sci., 59, 1550–1565.
\item
Coceal, O. and S. E. Belcher, 2005: A canopy model of mean winds through urban areas. Quart.
J. Roy. Meteor. Soc., 130, 1349–1372.
\item
Cuxart, J., P. Bougeault, and J.-L. Redelsperger, 2000: A turbulence scheme allowing for
mesoscale and large-eddy simulations. Quart. J. Roy. Meteor. Soc., 116, 1–30.
\item
Deardorff, J., 1978: Efficient prediction of ground temperature and moisture with inclusion of
a layer of vegetation. J. Geophys. Res., 83, 1889–1903.
\item
Geleyn, J.-F., 1988: Interpolation of wind, temperature and humidity values from model levels
to the height of measurement. Tellus, 40A, 347–351.
\item
Grini, A., P. Tulet, and L. Gomes, 2006: Dusty weather forecast using the mesonh atmospheric
model. J. Geophys. Res., 111, D19 205, doi:10.1029/2005JD007007.
\item
Hamdi, R. and V. Masson, 2008: Inclusion of a drag approach in the town energy balance
(teb) scheme: offline 1-d validation in a street canyon. Journal of Applied Meteorology and
Climatology, in press.
\item
H¨ogstr¨om, U., 1988: Non-dimensional wind and temperature profiles in the atmospheric surface
layer: A re-evaluation. Boundary-Layer Meteorol., 42, 680–687.
Kanda, M. and M. Hino, 1994: Organized structures in developing turbulent flow within and
above a plant canopy, using a large eddy simulation. Boundary-Layer Meteorol., 68, 237–
257.
\item
Kondo, H., Y. Genchi, Y. Kikegawa, Y. Ohashi, H. Yoshikado, and H. Komiyama, 2005: Development
of a multi-layer urban canopy model for the analysis of energy consumption in
a big city: Structure of the urban canopy model and its basic performance. Boundary-Layer
Meteorology, 116, 395–421, doi:doi:10.1007/s10546-005-0905-5.
\item
Lafore, J.P., J. Stein, N. Asencio, P. Bougeault, V. Ducrocq, J. Duron, C. Fischer, P. Hreil,
P. Mascart, V. Masson, J.P. Pinty, J.L. Redelsperger, E. Richard, and J. Vila-Guerau de Arellano,
1998: The Mso-NH atmospheric simulation system. Part I : Adiabatic formulation and
control simulation. Ann. Geophys., 16, 90–109.
\item
Lee, H. N., 1997: Improvement of surface flux calculations in the atmospheric surface layer. J.
Appl. Meteorol., 36, 1416–1423.
\item
Manabe, S., 1965: Climate and the ocean circulation 1. the atmospheric circulation and the
hydrology of the earth’s surface. Mon. Wea. Rev., 97, 739–774.
\item
Martilli, A., A. Clappier, and M. Rotach, 2002: An urban Surface exchange Parameterization
for mesoscale Models. Boundary-Layer Meteorol., 104, 261–304.
\item
Masson, V., 2006: Urban surface modeling and the meso-scale impact of cities. Theoretical and
Appl. Climatology, 84, 35–45.
\item
Masson, V., J.L. Champeaux, F. Chauvin, C. Meriguet, and R. Lacaze, 2003: A global data
base of land surface parameters at 1 km resolution in meteorological and climate models. J.
of Climate, 16 (9), 1261–1282.
\item
Noilhan, J. and S. Planton, 1989: A simple parameterization of land surface processes for
meteorological models. Mon. Wea. Rev., 117, 536–549.
\item
Park, H. and S. Hattori, 2004: Modeling scalar and heat sources, sinks, and fluxes
within a forest canopy during and after rainfall events. J. Geophys. Res., 109, D14 301,
dOI:10.1029/2003JD004360.
\item
Patton, E. G., K. J. Davis, M. C. Barth, and P. P. Sullivan, 2001: Decaying scalars emitted by a
forest canopy - a numerical study. Boundary-Layer Meteorol., 100, 91–129.
\item
Patton, E. G., P. P. Sullivan, and C.-H. Moeng, 2003: The influence of a forest canopy on topdown
and bottom-up diffusion in the planetary boundary layer. Quart. J. Roy. Meteor. Soc.,
129, 1415–1434.
\item
Paulson, C. A., 1970: The mathematical representation of wind and temperature profiles in the
unstable atmospheric surface layer. J. Appl. Meteorol., 9, 857–861.
\item
Pinty, J.-P. and P. Jabouille, 1998: A mixed-phase cloud parameterization for use in a mesoscale
non-hydrostatic model: Simulations of a squall line and of orographic precipitation. Conf. on
Cloud Physics, Everett, WA, AMS, 217–220.
\item
Pleim, J. E., 2006: A simple, efficient solution of flux-profile relationships in the atmospheric
surface layer. J. Appl. Meteorol. Clim., 45, 341–347.
\item
Polcher, J. and coauthors, 1998: A proposal for a general interface between land-surface
schemes and general circulation models. Global Planet. change, 19, 263–278.
\item
Redelsperger, J.-L., F. Mahe, and P. Carlotti, December 2001: A simple and general subgrid
model suitable both for surface layer and free-stream turbulence. Boundary-Layer Meteorology,
101, 375–408.
\item
Robock, A., K. Y. Vinikov, C. A. Schlosser, N. A. Speranskaya, and Y. Xue, 1995: Use of midlatitude
soil moisture and meteorological observations to validate soil moisture simulations
with biosphere and bucket models. Journal of Climate, 8, 15–35.
\item
Rotach, M. W., V. R., B. D., B. E., C. A., C. A., F. B., G. S.-E., M. G., M. H., M. V., O. T. R.,
P. E., R. H., R. M., R. Y.-A., R. D., S. J.-A., S. M., and V. J. A., 2005: Bubble - an urban
boundary layer meteorology project. Theoretical and Applied Climatology, 81, 231–261, dOI
10,1007/s00704-004-0117-9.
\item
Shaw, R. and U. Schumann, 1992: Large-eddy simulation of turbulent flow above and within a
forest. Boundary-Layer Meteorol., 61, 119–131.
\item
Shen, S. and M. Y. Leclerc, 1997: Modelling the turbulence structure in the canopy layer.
Agricultural and Forest Meteorology, 87, 1–84.
\item
Simon, E., F. X. Meixmer, L. Ganzeveld, and J. Kesselmeier, 2005: Coupled carbon-water
exchange of the amazon rain forest, 1. model description, parameterization and sensitivity
analysis. Biogeosciences Discussions, 2, 333–397.
\item
Stull, R. B., 1988: An introduction to Boundary Layer Meteorology. Kluwer, 666pp.
\item
Suna, H., T. L. Clarka, R. B. Stulla, and T. A. Black, 2006: Two-dimensional simulation of airflow and carbon dioxide transport over a forested mountain part i: Interactions between
thermally-forced circulations. Agricultural and Forest Meteorology, 140, 338–351,
doi:10.1016/j.agrformet.2006.03.023.
\item
Tulet, P., V. Crassier, F. Solmon, D. Guedalia, and R. Rosset, 2003: Description of the mesoscale
nonhydrostatic chemistry model and application to a transboundary pollution episode between
northern france and southern england. J. Geophys. Res., 108, D1, 4021.
\item
Xinmin, Z., Z. Ming, S. Bingkai, and W. Hanjie, 1999: Study on a boundary-layer numerical
model with inclusion of heterogeneous multi-layer vegetation. Advances in Atmospheric
Sciences, 16, 431–442, doi:10.1007/s00376-999-0021-4.

\end{description}

%%%%%%%%%%%%%%%%%%%%%%%%%%%%%%%%%%%%%%%%%%%%%%%%%%%
\section{1D TKE Oceanic model}
\begin{description}

\item
Belamari, S., 2005 : Report on uncertainty estimates of an optimal bulk formulation for surface turbulent
fluxes. MERSEA IP Deliverable, D.4.1.2, 29.
\item
Bougeault, P. and P. Lacarr\`ere, 1989 : Parameterization of orography-induced turbulence in a mesobeta
scale model. Mon. Wea. Rev., 117, 1872–1890.
\item
Brunke, M. A., C. W. Fairall, X. Zeng, L. Eymard, and J. A. Curry, 2003 : Which bulk aerodynamic
algorithms are least problematic in computing ocean surface turbulent fluxes ? J. Climate, 16, 619–
635.
\item
Businger, J. A., J. C. Wyngaard, Y. Izumi, and E. Bradley, 1971 : Flux profile relationship in the
atmospheric surface layer. J. Atmos. Sci., 28, 181–189.
\item
Charnock, H., 1955 : Wind stress on a water surface. Quart. J. Roy. Meteor. Soc., 81, 639–640.
\item
Fairall, C.W., E. F. Bradley, J. S. Godfrey, G. A.Wick, J. B. Edson, and G. S. Young, 1996a : Cool-skin
and warm-layer effects on sea surface temperature. J. Geophys. Res., 101, 1295–1308.
\item
Fairall, C. W., E. F. Bradley, J. E. Hare, A. A. Grachev, and J. B. Edson, 2003 : Bulk parameterization
of air-sea fluxes : Updates and verification for the COARE algorithm. J. Climate, 16, 571–591.
\item
Fairall, C.W., E. F. Bradley, D. P. Rogers, J. B. Edson, and G. S. Young, 1996b : Bulk parameterization
of air-sea fluxes for Tropical Ocean-Global Atmosphere Coupled-Ocean Atmosphere Response
Experiment. J. Geophys. Res., 101, 3747–3764.
\item
Gaspar, P., Y. Gr\'egoris, and J.-M. Lefevre, 1990 : A simple Eddy Kinetic Energy model for simulations
of the oceanic vertical mixing : Tests at station Papa and Long-Term Upper Ocean Study site. J.
Geophys. Res., 95, 16179–16193.
\item
Gosnell, R., C. Fairall, and P. J. Webster, 1995 : The sensible heat of rainfall in the tropical ocean. J.
Geophys. Res., 100, 18437–18442.
\item
Grachev, A. A. and C. W. Fairall, 1997 : Dependence of the Monin-Obukhov stability parameter on
the bulk richardson number over the ocean. J. Appl. Meteor., 36, 406–414.
Hare, J. E., P. O. G. Persson, C.W. Fairall, and J. B. Edson, 1999 : Behavior of Charnock’s relationship
for high wind conditions. 13th Symp. on Boundary Layers and Turbulence, Amer. Meteor. Soc.,
Dallas, TX, 252–255.
\item
Kraus, E. B., 1972 : Atmosphere-ocean interactions. London Oxford University press.
\item
Lebeaupin Brossier, C., 2007 : \'etude du couplage oc\'ean-atmosph\`ere associ\'e aux \'episodes de pluie
intenses en r\'egion M\'editerran\'eenne. Ph.D. thesis, Univ. P. Sabatier, Toulouse III, 228 pp.
\item
Lebeaupin Brossier, C., V. Ducrocq, and H. Giordani, 2008a : Sensitivity of threeMediterranean heavy
rain events to two different sea surface fluxes parameterization in high-resolution numerical modelling.
J. Geophys. Res., in revision.

\end{description}

%%%%%%%%%%%%%%%%%%%%%%%%%%%%%%%%%%%%%%%%%%%%%%%%%%%
\section{Flake}
\begin{description}

\item
Beyrich, F., J.-P. Leps, M. Mauder, J. Bange, T. Foken, S. Huneke, H. Lohse, A. Luedi, W. M. L. Meijninger, D. Mironov, U. Weisensee, and P. Zittel, 2006: Area-averaged surface fluxes over the LITFASS region based on eddy-covariance measurements. Boundary-Layer Meteorol., 121, 33-65. doi:10.1007/s10546-006-9052-x 
\item
Braslavski, D., 2004: Experiments with Zero-Dimensional Lake Model. M.Sc. Thesis, Russian State Hydrometeorological University, St. Petersburg, Russia, 51 pp. (in Russian) 
\item
Dutra, E., V. Stepanenko, P. A. Miranda, P. Viterbo, D. Mironov, and V. N. Lykosov, 2006: Evaporation and seasonal temperature changes in lakes of the Iberian Peninsula. Proc. of the 5th Portuguese-Spanish Assembly of Geophysics and Geodesy, 30 January - 3 February 2006, Sevilha, Spain, 4 pp. (PDF)
\item
Ganbat, G. O., 2006: External-Parameter Fields for Hydrodynamic Models of the Atmosphere. B.Sc. Thesis, Russian State Hydrometeorological University, St. Petersburg, Russia, 36 pp. (in Russian) (PDF) 
\item
Golosov, S., O. Maher, E. Schipunova, A. Terzhevik, G. Zdorovennova, and G. Kirillin, 2006: Physical background of the development of oxygen depletion in ice-covered lakes. Oecologia, 151, 331-340. doi: 10.1007/s00442-006-0543-8 
\item
Golosov, S. G., A. Terzhevik, O. A. Maher, E. Shipunova, and G. Zdorovennova, 2004: Modelling seasonal dynamics of dissolved oxygen in a shallow stratified lake. Proc. of the 8th Workshop on Physical Processes in Natural Waters, L. Bengtsson and O. A. Maher, Eds., University of Lund, Lund, Sweden, 153-164.   
\item
Kirillin, G., 2003: Modelling of the shallow lake response to climate variability. Proc. of the 7th Workshop on Physical Processes in Natural Waters, A. Yu. Terzhevik, Ed., Northern Water Problems Institute, Russian Academy of Sciences, Petrozavodsk, Karelia, Russia, 144-148. (PDF)
\item
Kourzeneva K., and D. Braslavsky, 2005: Lake model FLake, coupling with atmospheric model: first steps. Proc. of the 4th SRNWP/HIRLAM Workshop on Surface Processes and Assimilation of Surface Variables jointly with HIRLAM Workshop on Turbulence, S. Gollvik, Ed., 15-17 September 2004, SMHI, Norrkoping, Sweden, 43-54. (PDF) 
\item
Maher, O. A., S. Golosov, A. Mitrokhov, N. Palshin, M. Petrov, E. Shipunova, A. Terzhevik, R. Zdorovennov, and G. Zdorovennova, 2004: Dynamics of Dissolved Oxygen in a Shallow Lake: Measurements and Modelling. Department of Water Resources Engineering, Institute of Technology, University of Lund, Report 3247, Lund, 51 pp.
\item
Martynov, A., R. Laprise, and L. Sushama, 2008: Off-line lake water and ice simulations: a step towards the interactive lake coupling with the Canadian Regional Climate Model. Geophysical Research Abstracts, Vol. 10, EGU2008-A-02898. (PDF)
\item
Mironov, D. V., 1991: Air-water interaction parameters over lakes. Modeling Air-Lake interaction. Physical Background, S. S. Zilitinkevich, Ed., Springer-Verlag, Berlin, etc., 50-62.
\item
Mironov, D. V., 2008: Parameterization of lakes in numerical weather prediction. Description of a lake model. COSMO Technical Report, No. 11, Deutscher Wetterdienst, Offenbach am Main, Germany, 41 pp. (PDF)
\item
Mironov, D., S. Golosov, E. Heise, E. Kourzeneva, B. Ritter, N. Scheider, and A. Terzhevik, 2005a: FLake - A Lake Model for Environmental Applications. Proc. of the 9th Workshop on Physical Processes in Natural Waters, 4 - 6 September 2005, A. Folkard and I. Jones, Eds., Lancaster University, UK, 73. (PDF)
\item
Mironov, D. V., S. D. Golosov, S. S. Zilitinkevich, K. D. Kreiman, and A. Yu. Terzhevik, 1991: Seasonal changes of temperature and mixing conditions in a lake. Modeling Air-Lake interaction. Physical Background, S. S. Zilitinkevich, Ed., Springer-Verlag, Berlin, etc., 74-90. 
\item
Mironov, D. V., S. D. Golosov, and I. S. Zverev, 2003a: Temperature profile in lake bottom sediments: an analytical self-similar solution. Proc. of the 7th Workshop on Physical Processes in Natural Waters, A. Yu. Terzhevik, Ed., Northern Water Problems Institute, Russian Academy of Sciences, Petrozavodsk, Karelia, Russia, 90-97. (PDF)
\item
Mironov, D., E. Heise, E. Kourzeneva, B. Ritter, and N. Schneider, 2007: Parameterisation of lakes in numerical weather prediction and climate models. Proc. of the 11th Workshop on Physical Processes in Natural Waters, L. Umlauf and G. Kirillin, Eds., Berichte des IGB, Vol. 25, Berlin, Germany, 101-108. (PDF,  download the entire volume)
\item
Mironov, D., G. Kirillin, E. Heise, S. Golosov, A. Terzhevik, and I. Zverev, 2003b: Parameterization of lakes in numerical models for environmental applications. Proc. of the 7th Workshop on Physical Processes in Natural Waters, A. Yu. Terzhevik, Ed., Northern Water Problems Institute, Russian Academy of Sciences, Petrozavodsk, Karelia, Russia, 135-143. (PDF) 
\item
Mironov, D., and B. Ritter, 2003: A first version of the ice model for the global NWP system GME of the German Weather Service. Research Activities in Atmospheric and Oceanic Modelling, J. Cote, Ed., Report No. 33, April 2003, WMO/TD, 4.13-4.14. (PDF)
\item
Mironov, D., and B. Ritter, 2004a: A New Sea Ice Model for GME. Technical Note, Deutscher Wetterdienst, Offenbach am Main, Germany, 12 pp. (PS) 
\item
Mironov, D., and B. Ritter, 2004b: Testing the new ice model for the global NWP system GME of the German Weather Service. Research Activities in Atmospheric and Oceanic Modelling, J. Cote, Ed., Report No. 34, April 2004, WMO/TD-No. 1220, 4.21-4.22.  (PDF) 
\item
Mironov, D., and B. Ritter, 2007: A thermodynamic sea ice model for the global numerical weather prediction system GME of the German Weather Service. Submitted to Mon. Weather. Rev.
\item
Mironov, D., N. Schneider, B. Ritter, and E. Heise, 2005b: Implementation of a Lake Model FLake into the Limited-Area NWP System LM of the German Weather Service: Preliminary Results. Research Activities in Atmospheric and Oceanic Modelling, J. Cote, Ed., Report No. 35, Month 2005, WMO/TD-No. 1220, 4.15-4.16. (PDF, username science, password science)
\item
Mironov, D., A. Terzhevik, F. Beyrich, and E. Heise, 2004: A Lake Model for Use in Numerical Weather Prediction Systems. Research Activities in Atmospheric and Oceanic Modelling, J. Cote, Ed., Report No. 34, April 2004, WMO/TD-No. 1220, 4.23-4.24. (PDF, username science, password science)
\item
Mironov, D., A. Terzhevik, F. Beyrich, E. Heise, and H. Lohse, 2003c: A two-layer lake model for use in numerical weather prediction. Proc. of the Baltic HIRLAM Workshop, 17-20 November 2003, St. Petersburg, Russia, 83-85. (PDF) 
\item
Mironov, D., A. Terzhevik, E. Heise, and F. Beyrich, 2003d: Parameterization of lakes in NWP: description of a lake model and single-column tests. Fifth International SRNWP-Workshop on Nonhydrostatic Modelling. Abstracts, Deutscher Wetterdienst, Geschaeftsbereich Forschung und Entwicklung, Arbeitsergebnisse Nr. 78, E. Heise and J. Steppeler, Eds., October 2003, Offenbach am Main, Germany.
\item
Petrov, M., A. Terzhevik, N. Palshin, R. Zdorovennov, and G. Zdorovennova, 2005: Absorption of solar radiation by snow-and-ice cover of lakes. Vodnye Resursy, 32, 546-554. (in Russian; English translation: Water Resources, p. 496-504)
\item
Petrov, M., A. Terzhevik, R. Zdorovennov, and G. Zdorovennova, 2006: The thermal structure of a shallow lake in early winter. Vodnye Resursy (Water Resources), 33, 154-162. (in Russian; English translation: Water Resources, p. 135-143)
\item
Petrov, M., A. Terzhevik, R. Zdorovennov, and G. Zdorovennova, 2007: Water motions in a shallow ice-covered lake. Vodnye Resursy (Water Resources), 34, 113-122.
\item
Stepanenko, V. M., 2007: Numerical Modelling of the Interaction of the Atmosphere with Inland Water Bodies. Ph.D. Thesis, Moscow State University, Scientific Research Computing Center, Moscow, Russia, 159 pp. (in Russian) (PDF, DJVU) (Abstract of thesis, 32 pp., in Russian, PDF) 
\item
Wilberforce, K., 2006: Lake Parameterisation in NWP and Climate Models. B.Sc. Thesis, Russian State Hydrometeorological University, St. Petersburg, Russia, 23 pp. 

\end{description}

%%%%%%%%%%%%%%%%%%%%%%%%%%%%%%%%%%%%%%%%%%%%%%%%%%%
\section{Chemistry}
\begin{description}

\item
Alfaro, S., and L. Gomes, Modeling mineral aerosol production by wind erosion: Emission intensities and
  aerosol size distributions in source areas, J. Geophys. Res., 106(D16), 18,075–18,084, 2001.
\item
Baer, M., and K. Nester, Parametrization of trace gas dry deposition velocities for a regional mesoscale
  diffusion model, Ann. Geophys., 10, 1992.
\item
Baldocchi, D., B. Hicks, and P. Camara, A canopy stomatal resistance model for gaseous deposition to
  vegetated surfaces, Atmos. Environ., 21, 1987.
\item
Blanchard, D. C., The production, distribution, and bacterial enrichment of the sea-salt aerosol. air-sea
  exchange of gases and particles, p. s. liss andw. g. n. slinn, eds., d. reidel, Norwell, Mass., pp. 407–454,
  1983.
\item
Erisman, J., and D. Baldocchi, Modelling dry deposition of so2 , Tellus, 46B, 159–171, 1994.
\item
Fecan, F., B. Marticorena, and G. Bergametti, Parameterization due to the increase of the aeolian erosion
  threshold wind friction velocity due to soil moisture for arid and semi-arid areas, Annales Geophysicae,
  17, 149–157, 1999.
\item
Ganzeveld, L., and J. Lelieveld, Dry deposition parametrization in a chemistry general circulation model
  and its influence on the distribution of reactive trace gases., J. Geophys. Res., 100, 20,999–21,012, 1995.
\item
Garrat, J., and B. B. Hicks, Momentum, heat and water vapor transfer to and from natural and artificial
  surfaces., Quart. J. Roy. Meteor. Soc., 99, 680–687, 1973.
\item
Grini, A., G. Myhre, C. Zender, and I. Isaksen, Model simulation of dust sources and transport
  in the global atmosphere. effects of soil erodibility and wind speed variability, J. Geophys. Res.,
  doi:10.1029/DOO4JD005037, 2005.
\item
Hicks, B., D. Badolcchi, T. Meyers, R. Hoskers, and D. Matt, A preliminary multiple resistance routine for
  deriving dry deposition velocities from measured quantities, Water Air Soil Pollu., 36, 311–330, 1987.
Iversen, J., and B. White, Saltation threshold on earth, mars and venus, Sedimentology, 29, 111–119, 1982.
Kondo, J., and H. Yamazawa, Measurement of snow surface emissivity., Bound. Layer. Meteor., 34, 415–
  416, 1986.
\item
Laurent, B., B. Marticorena, G. Bergametti, P. Chazette, F. Maignan, and C. Schmechtig, Simulation of
  the mineral dust emission frequencies from desert areas of china and mongolia using an aerodynamic
  roughness length map derived from polder/adeos 1 surface products, J. Geophys. Res., 110, D18S04,
  doi:10.1029/2004JD005013, 2005.
\item
Marticorena, B., and G. Bergametti, Modeling of the atmospheric dust cycle: 1. design of a soil derived dust
  emission scheme, J. Geophys. Res., 100, 16,415–16,429, 1995.
\item
Masson, V., A physically-based scheme for the urban energy balance in atmospheric models, Boundary-
  Layer Meteorology, 94, 357–397, 2000.
\item
Meyers, T. P., and D. D. Baldocchi, A comparison of models for deriving dry deposition fluxes of o3 and
  so2 to a forest canopy., Tellus, 40B, 270–284, 1988.
\item
Monahan, E. C., C. W. Fairall, K. L. Davidson, and P. J. Boyle, Observed inter-relations between 10 m
  winds, ocean whitecaps and marine aerosols., Quart. J. Roy. Meteor. Soc., 109, 379–392, 1983.
\item
Muller, H., F. Meixner, G. Kramm, D. Fowler, G. J. Dollard, and M. Possanzini, Determination of hno3
  deposition by modified bowen ratio and aerodynamic profile techniques., Tellus, 45B, 346–367, 1993.
\item
Schulz, M., G. deLeeuw, and Y. Balkanski, Sea-salt aerosol source functions and emissions. emissions of
  atmospheric trace compounds, P.A.C. Granier and C.E. Reeves, Eds., Kluwer, pp. 333–359, 2004.
\item
Sehmel, G., Particle and gas dry deposition: a review., Atmos. Environ., 14, 981–1011, 1980.
\item
Seinfeld, J., and S. Pandis, Atmospheric Chemistry and Physics, Wiley interscience pub, 1997.
\item
Shao, Y., A model for mineral dust erosion, J. Geophys. Res., 106(D17), 20,239–20,254, 2001.
\item
Shao, Y., M. Raupach, and P.A.Findlater, The effect of saltation bombardment on the entrainment of dust
  by wind., J. Geophys. Res., 98, 12,719–12,726, 1993.
\item
Sheih, C., M. Wesely, and B. Hicks, Estimated dry deposition velocities of sulfur over the eastern united
  states and surrounding regions, Atmos. Environ., 13, 1979.
\item
Solmon, F., C. Sarrat, D. Serca, P. Tulet, and R. Rosset, Isoprene and monoterpenes biogenic emissions in
  france: modeling and impact during a regional pollution episode, Atmos. Environ., 38, 3853–3865, 2004.
\item
VanPul, W. A. J., and A. F. G. Jacobs, The conductance of a maize crop and the underlying soil to ozone
  under various environmental conditions., Bound. Layer. Meteor., 63, 83–99, 1994.
\item
Vignati, E., G. DeLeeuw, and R. Berkowicz, Modeling coastal aerosol transport and effects of surf-produced
  aerosols on processes in the marine atmospheric boundary layer., J. Geophys. Res., 106(D17), 20,225–
  20,238, 2001.
\item
Walcek, C., R. Brost, J. Chang, and M. Wesely, So2 , sulfate and hno3 deposition velocities computed using
  regional landuse and meteorological data, Atmos. Environ., 20, 949–964, 1996.
\item
Walmsley., J., and M. Wesely, Modification of coded parametrizations of surface resistances to gaseous dry
  deposition, Atmos. Environ., 30, 1181–1188, 1996.
\item
Wesely, M., Parametrizations of surface resistance to gaseous dry deposition in regional scale, numerical
  models, Atmos. Environ., 23, 1293–1304, 1989.
\item
Wesely, M., and B. Hicks, Some factors that affect the deposition rates of sulfur dioxide and similar gases
  on vegetation, J. Air. Control. Assoc., 27, 1110–1116, 1977.
\item
Zender, C., H. Bian, and D. Newman, The mineral dust entrainment and deposition (dead) model: Descrip-
  tion and global dust distribution, J. Geophys. Res., 108(D14), 4416, http://dust.ess.uci.edu/dead/, 2003.
\item
Zender, C., R. Miller, and I. Tegen, Quantifying mineral dust mass budgets: Terminology, constraints, and
  current estimate, Eos Trans, 85(48), 509–512, 2004.

\end{description}


